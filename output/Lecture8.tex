\documentclass[12pt,a4paper]{article}

\usepackage[a4paper,text={16.5cm,25.2cm},centering]{geometry}
\usepackage{lmodern}
\usepackage{amssymb,amsmath}
\usepackage{bm}
\usepackage{graphicx}
\usepackage{microtype}
\usepackage{hyperref}
\setlength{\parindent}{0pt}
\setlength{\parskip}{1.2ex}

\hypersetup
       {   pdfauthor = { Sheehan Olver },
           pdftitle={ foo },
           colorlinks=TRUE,
           linkcolor=black,
           citecolor=blue,
           urlcolor=blue
       }




\usepackage{upquote}
\usepackage{listings}
\usepackage{xcolor}
\lstset{
    basicstyle=\ttfamily\footnotesize,
    upquote=true,
    breaklines=true,
    breakindent=0pt,
    keepspaces=true,
    showspaces=false,
    columns=fullflexible,
    showtabs=false,
    showstringspaces=false,
    escapeinside={(*@}{@*)},
    extendedchars=true,
}
\newcommand{\HLJLt}[1]{#1}
\newcommand{\HLJLw}[1]{#1}
\newcommand{\HLJLe}[1]{#1}
\newcommand{\HLJLeB}[1]{#1}
\newcommand{\HLJLo}[1]{#1}
\newcommand{\HLJLk}[1]{\textcolor[RGB]{148,91,176}{\textbf{#1}}}
\newcommand{\HLJLkc}[1]{\textcolor[RGB]{59,151,46}{\textit{#1}}}
\newcommand{\HLJLkd}[1]{\textcolor[RGB]{214,102,97}{\textit{#1}}}
\newcommand{\HLJLkn}[1]{\textcolor[RGB]{148,91,176}{\textbf{#1}}}
\newcommand{\HLJLkp}[1]{\textcolor[RGB]{148,91,176}{\textbf{#1}}}
\newcommand{\HLJLkr}[1]{\textcolor[RGB]{148,91,176}{\textbf{#1}}}
\newcommand{\HLJLkt}[1]{\textcolor[RGB]{148,91,176}{\textbf{#1}}}
\newcommand{\HLJLn}[1]{#1}
\newcommand{\HLJLna}[1]{#1}
\newcommand{\HLJLnb}[1]{#1}
\newcommand{\HLJLnbp}[1]{#1}
\newcommand{\HLJLnc}[1]{#1}
\newcommand{\HLJLncB}[1]{#1}
\newcommand{\HLJLnd}[1]{\textcolor[RGB]{214,102,97}{#1}}
\newcommand{\HLJLne}[1]{#1}
\newcommand{\HLJLneB}[1]{#1}
\newcommand{\HLJLnf}[1]{\textcolor[RGB]{66,102,213}{#1}}
\newcommand{\HLJLnfm}[1]{\textcolor[RGB]{66,102,213}{#1}}
\newcommand{\HLJLnp}[1]{#1}
\newcommand{\HLJLnl}[1]{#1}
\newcommand{\HLJLnn}[1]{#1}
\newcommand{\HLJLno}[1]{#1}
\newcommand{\HLJLnt}[1]{#1}
\newcommand{\HLJLnv}[1]{#1}
\newcommand{\HLJLnvc}[1]{#1}
\newcommand{\HLJLnvg}[1]{#1}
\newcommand{\HLJLnvi}[1]{#1}
\newcommand{\HLJLnvm}[1]{#1}
\newcommand{\HLJLl}[1]{#1}
\newcommand{\HLJLld}[1]{\textcolor[RGB]{148,91,176}{\textit{#1}}}
\newcommand{\HLJLs}[1]{\textcolor[RGB]{201,61,57}{#1}}
\newcommand{\HLJLsa}[1]{\textcolor[RGB]{201,61,57}{#1}}
\newcommand{\HLJLsb}[1]{\textcolor[RGB]{201,61,57}{#1}}
\newcommand{\HLJLsc}[1]{\textcolor[RGB]{201,61,57}{#1}}
\newcommand{\HLJLsd}[1]{\textcolor[RGB]{201,61,57}{#1}}
\newcommand{\HLJLsdB}[1]{\textcolor[RGB]{201,61,57}{#1}}
\newcommand{\HLJLsdC}[1]{\textcolor[RGB]{201,61,57}{#1}}
\newcommand{\HLJLse}[1]{\textcolor[RGB]{59,151,46}{#1}}
\newcommand{\HLJLsh}[1]{\textcolor[RGB]{201,61,57}{#1}}
\newcommand{\HLJLsi}[1]{#1}
\newcommand{\HLJLso}[1]{\textcolor[RGB]{201,61,57}{#1}}
\newcommand{\HLJLsr}[1]{\textcolor[RGB]{201,61,57}{#1}}
\newcommand{\HLJLss}[1]{\textcolor[RGB]{201,61,57}{#1}}
\newcommand{\HLJLssB}[1]{\textcolor[RGB]{201,61,57}{#1}}
\newcommand{\HLJLnB}[1]{\textcolor[RGB]{59,151,46}{#1}}
\newcommand{\HLJLnbB}[1]{\textcolor[RGB]{59,151,46}{#1}}
\newcommand{\HLJLnfB}[1]{\textcolor[RGB]{59,151,46}{#1}}
\newcommand{\HLJLnh}[1]{\textcolor[RGB]{59,151,46}{#1}}
\newcommand{\HLJLni}[1]{\textcolor[RGB]{59,151,46}{#1}}
\newcommand{\HLJLnil}[1]{\textcolor[RGB]{59,151,46}{#1}}
\newcommand{\HLJLnoB}[1]{\textcolor[RGB]{59,151,46}{#1}}
\newcommand{\HLJLoB}[1]{\textcolor[RGB]{102,102,102}{\textbf{#1}}}
\newcommand{\HLJLow}[1]{\textcolor[RGB]{102,102,102}{\textbf{#1}}}
\newcommand{\HLJLp}[1]{#1}
\newcommand{\HLJLc}[1]{\textcolor[RGB]{153,153,119}{\textit{#1}}}
\newcommand{\HLJLch}[1]{\textcolor[RGB]{153,153,119}{\textit{#1}}}
\newcommand{\HLJLcm}[1]{\textcolor[RGB]{153,153,119}{\textit{#1}}}
\newcommand{\HLJLcp}[1]{\textcolor[RGB]{153,153,119}{\textit{#1}}}
\newcommand{\HLJLcpB}[1]{\textcolor[RGB]{153,153,119}{\textit{#1}}}
\newcommand{\HLJLcs}[1]{\textcolor[RGB]{153,153,119}{\textit{#1}}}
\newcommand{\HLJLcsB}[1]{\textcolor[RGB]{153,153,119}{\textit{#1}}}
\newcommand{\HLJLg}[1]{#1}
\newcommand{\HLJLgd}[1]{#1}
\newcommand{\HLJLge}[1]{#1}
\newcommand{\HLJLgeB}[1]{#1}
\newcommand{\HLJLgh}[1]{#1}
\newcommand{\HLJLgi}[1]{#1}
\newcommand{\HLJLgo}[1]{#1}
\newcommand{\HLJLgp}[1]{#1}
\newcommand{\HLJLgs}[1]{#1}
\newcommand{\HLJLgsB}[1]{#1}
\newcommand{\HLJLgt}[1]{#1}



\def\qqand{\qquad\hbox{and}\qquad}
\def\qqfor{\qquad\hbox{for}\qquad}
\def\D{ {\rm d} }
\def\I{ {\rm i} }
\def\E{ {\rm e} }
\def\C{ {\mathbb C} }
\def\R{ {\mathbb R} }
\def\CC{ {\cal C} }
\def\HH{ {\cal H} }
\def\vc#1{ {\mathbf #1} }
\def\bbC{ {\mathbb C} }

\def\qqqquad{\qquad\qquad}
\def\qqfor{\qquad\hbox{for}\qquad}
\def\qqwhere{\qquad\hbox{where}\qquad}
\def\Res_#1{\underset{#1}{\rm Res}\,}
\def\sech{ {\rm sech}\, }



\def\Xint#1{ \mathchoice
   {\XXint\displaystyle\textstyle{#1} }%
   {\XXint\textstyle\scriptstyle{#1} }%
   {\XXint\scriptstyle\scriptscriptstyle{#1} }%
   {\XXint\scriptscriptstyle\scriptscriptstyle{#1} }%
   \!\int}
\def\XXint#1#2#3{ {\setbox0=\hbox{$#1{#2#3}{\int}$}
     \vcenter{\hbox{$#2#3$}}\kern-.5\wd0} }
\def\ddashint{\Xint=}
\def\dashint{\Xint-}
% \def\dashint
\def\infdashint{\dashint_{-\infty}^\infty}




\def\addtab#1={#1\;&=}
\def\ccr{\\\addtab}
\def\ip<#1>{\left\langle{#1}\right\rangle}
\def\dx{\D x}
\def\dt{\D t}
\def\dz{\D z}

\def\norm#1{\left\| #1 \right\|}

\def\pr(#1){\left({#1}\right)}
\def\br[#1]{\left[{#1}\right]}

\def\abs#1{\left|{#1}\right|}
\def\fpr(#1){\!\pr({#1})}

\def\sopmatrix#1{ \begin{pmatrix}#1\end{pmatrix} }

\def\endash{–}
\def\mdblksquare{\blacksquare}

\begin{document}

\textbf{M3M6: Methods of Mathematical Physics}

Dr. Sheehan Olver

s.olver@imperial.ac.uk

\section{Lecture 8: Matrix  functions via Cauchy's integral formula}
\subsubsection{Application: calculating matrix exponentials}
Let $A \in {\mathbb C}^{d \times d}$,   ${\mathbf u}_0 \in {\mathbb C}^d$ and consider the constant coefficient linear ODE

\[
    {\mathbf u}'(t) = A {\mathbf u}(t)\qquad\hbox{and}\qquad {\mathbf u}(0) = {\mathbf u}_0(0)
\]
The solution to this is given by the \emph{matrix exponential}

\[
    {\mathbf u}(t) = \exp(A t) {\mathbf u}_0
\]
\begin{verbatim}
where the matrix exponential is defined by it's Taylor series:
\end{verbatim}
\[
    \exp(A) = \sum_{k=0}^\infty {A^k \over k!}
\]
This has stability problems, so a more convenient form is as follows:

\textbf{Theorem (Cauchy integral representation for matrix exponential)} Let $A \in {\mathbb C}^{n \times n}$ be a diagonalizable matrix:

\[
    A = V \Lambda V^{-1} \qquad\hbox{for}\qquad\Lambda =
    \begin{pmatrix}\lambda_1 \cr &\ddots \cr && \lambda_d\end{pmatrix}
\]
Let $\gamma$ be a contour that surrounds the spectrum of $A$. Then we have

\[
    \exp(A) = {1 \over 2 \pi i} \oint_\gamma e^z (z I - A)^{-1} dz
\]
\textbf{Proof} Note by definition


\begin{align*}
\exp(A) &= \sum_{k=0}^\infty {A^k \over k!} = V \sum_{k=0}^\infty {\Lambda^k \over k!}V^{-1} = V \begin{pmatrix} \E^{\lambda_1}\\ & \ddots \\ && \E^{\lambda_d} \end{pmatrix} V^{-1}  \\
&= {1 \over 2\pi \I} V \begin{pmatrix} \oint_\gamma {\E^\zeta \over \zeta - \lambda_1 } \D\zeta \\ & \ddots \\ && \oint_\gamma {\E^\zeta \over \zeta - \lambda_d } \D\zeta\end{pmatrix} V^{-1} 
\end{align*}
It was important here that $\gamma$ surrounded all $\zeta_j$.

We now take out the integration from the matrix, the easiest way to see this is to apply the Trapezium rule approximation:


\begin{align*}
V\begin{pmatrix} \oint_\gamma {\E^\zeta \over \zeta - \lambda_1 } \D\zeta \\ & \ddots \\ && \oint_\gamma {\E^\zeta \over \zeta - \lambda_d } \D\zeta\end{pmatrix} V^{-1} &= V\lim_{n \rightarrow \infty} \begin{pmatrix} \sum_{j=1}^n {w_j \E^\zeta_j \over \zeta_j - \lambda_1 }  \\ & \ddots \\ && \sum_{j=1}^n {w_j \E^\zeta_j \over \zeta_j - \lambda_d } \end{pmatrix} V^{-1} \\
&= \lim_{n \rightarrow \infty}  V\sum_{j=1}^n w_j \E^\zeta_j \begin{pmatrix}  {1 \over \zeta_j - \lambda_1 }  \\ & \ddots \\ &&  {1 \over \zeta_j - \lambda_d } \end{pmatrix} V^{-1} \\
&= \lim_{n \rightarrow \infty}  \sum_{j=1}^n w_j \E^\zeta_j V(I \zeta_j - \Lambda)^{-1}V^{-1} \\
&= \oint_\gamma \E^\zeta V(I \zeta - \Lambda)^{-1} V^{-1} \D \zeta 
= \oint_\gamma \E^\zeta (I \zeta - V\Lambda V^{-1})^{-1}  \D \zeta \\
& = \oint_\gamma \E^\zeta (I \zeta - A)^{-1}  \D \zeta \\
\end{align*}
\ensuremath{\blacksquare}

\emph{Demonstration} we use this formula alongside the complex trapezium rule to calculate matrix exponentials.  Begin by creating a random symmetric matrix (which only has real eigenvalues):


\begin{lstlisting}
(*@\HLJLk{using}@*) (*@\HLJLn{LinearAlgebra}@*)

(*@\HLJLn{A}@*) (*@\HLJLoB{=}@*) (*@\HLJLnf{randn}@*)(*@\HLJLp{(}@*)(*@\HLJLni{5}@*)(*@\HLJLp{,}@*)(*@\HLJLni{5}@*)(*@\HLJLp{)}@*)
(*@\HLJLn{A}@*) (*@\HLJLoB{=}@*) (*@\HLJLn{A}@*) (*@\HLJLoB{+}@*) (*@\HLJLn{A}@*)(*@\HLJLoB{{\textquotesingle}}@*)
(*@\HLJLn{\ensuremath{\lambda}}@*) (*@\HLJLoB{=}@*) (*@\HLJLnf{eigvals}@*)(*@\HLJLp{(}@*)(*@\HLJLn{A}@*)(*@\HLJLp{)}@*)
\end{lstlisting}

\begin{lstlisting}
5-element Array{Float64,1}:
 -3.161605416383148 
 -1.8023650248203766
  0.5534971606981841
  1.8458549349231879
  4.448031066956975
\end{lstlisting}


We can now by hand create a circle that surrounds all the eigenvalues:


\begin{lstlisting}
(*@\HLJLk{function}@*) (*@\HLJLnf{circle{\_}rule}@*)(*@\HLJLp{(}@*)(*@\HLJLn{n}@*)(*@\HLJLp{,}@*) (*@\HLJLn{r}@*)(*@\HLJLp{)}@*) 
    (*@\HLJLn{\ensuremath{\theta}}@*) (*@\HLJLoB{=}@*) (*@\HLJLnf{periodic{\_}rule}@*)(*@\HLJLp{(}@*)(*@\HLJLn{n}@*)(*@\HLJLp{)[}@*)(*@\HLJLni{1}@*)(*@\HLJLp{]}@*)
    (*@\HLJLn{r}@*)(*@\HLJLoB{*}@*)(*@\HLJLn{exp}@*)(*@\HLJLoB{.}@*)(*@\HLJLp{(}@*)(*@\HLJLn{im}@*)(*@\HLJLoB{*}@*)(*@\HLJLn{\ensuremath{\theta}}@*)(*@\HLJLp{),}@*) (*@\HLJLni{2}@*)(*@\HLJLn{\ensuremath{\pi}}@*)(*@\HLJLoB{*}@*)(*@\HLJLn{im}@*)(*@\HLJLoB{*}@*)(*@\HLJLn{r}@*)(*@\HLJLoB{/}@*)(*@\HLJLn{n}@*)(*@\HLJLoB{*}@*)(*@\HLJLn{exp}@*)(*@\HLJLoB{.}@*)(*@\HLJLp{(}@*)(*@\HLJLn{im}@*)(*@\HLJLoB{*}@*)(*@\HLJLn{\ensuremath{\theta}}@*)(*@\HLJLp{)}@*)
(*@\HLJLk{end}@*)
(*@\HLJLn{z}@*)(*@\HLJLp{,}@*)(*@\HLJLn{w}@*) (*@\HLJLoB{=}@*) (*@\HLJLnf{circle{\_}rule}@*)(*@\HLJLp{(}@*)(*@\HLJLni{100}@*)(*@\HLJLp{,}@*)(*@\HLJLnf{maximum}@*)(*@\HLJLp{(}@*)(*@\HLJLn{abs}@*)(*@\HLJLoB{.}@*)(*@\HLJLp{(}@*)(*@\HLJLn{\ensuremath{\lambda}}@*)(*@\HLJLp{))}@*)(*@\HLJLoB{+}@*)(*@\HLJLni{1}@*)(*@\HLJLp{)}@*)
\end{lstlisting}

\begin{lstlisting}
Error: UndefVarError: periodic_rule not defined
\end{lstlisting}


\begin{lstlisting}
(*@\HLJLnf{plot}@*)(*@\HLJLp{(}@*)(*@\HLJLn{z}@*)(*@\HLJLp{)}@*)
\end{lstlisting}

\begin{lstlisting}
Error: UndefVarError: plot not defined
\end{lstlisting}


\begin{lstlisting}
(*@\HLJLnf{scatter!}@*)(*@\HLJLp{(}@*)(*@\HLJLn{\ensuremath{\lambda}}@*)(*@\HLJLp{,}@*)(*@\HLJLnf{zeros}@*)(*@\HLJLp{(}@*)(*@\HLJLni{5}@*)(*@\HLJLp{);}@*) (*@\HLJLn{label}@*) (*@\HLJLoB{=}@*) (*@\HLJLs{"{}eigenvalues}@*) (*@\HLJLs{of}@*) (*@\HLJLs{A"{}}@*)(*@\HLJLp{)}@*)
\end{lstlisting}

\begin{lstlisting}
Error: UndefVarError: scatter! not defined
\end{lstlisting}


Here we wrap this up into a function \texttt{circle\_exp} that calculates the matrix exponential:


\begin{lstlisting}
(*@\HLJLk{function}@*) (*@\HLJLnf{circle{\_}exp}@*)(*@\HLJLp{(}@*)(*@\HLJLn{A}@*)(*@\HLJLp{,}@*) (*@\HLJLn{n}@*)(*@\HLJLp{,}@*) (*@\HLJLn{z\ensuremath{\_0}}@*)(*@\HLJLp{,}@*) (*@\HLJLn{r}@*)(*@\HLJLp{)}@*)
    (*@\HLJLn{z}@*)(*@\HLJLp{,}@*)(*@\HLJLn{w}@*) (*@\HLJLoB{=}@*) (*@\HLJLnf{circle{\_}rule}@*)(*@\HLJLp{(}@*)(*@\HLJLn{n}@*)(*@\HLJLp{,}@*)(*@\HLJLn{r}@*)(*@\HLJLp{)}@*)
    (*@\HLJLn{z}@*) (*@\HLJLoB{.+=}@*) (*@\HLJLn{z\ensuremath{\_0}}@*)

    (*@\HLJLn{ret}@*) (*@\HLJLoB{=}@*) (*@\HLJLnf{zero}@*)(*@\HLJLp{(}@*)(*@\HLJLn{A}@*)(*@\HLJLp{)}@*)
    (*@\HLJLk{for}@*) (*@\HLJLn{j}@*)(*@\HLJLoB{=}@*)(*@\HLJLni{1}@*)(*@\HLJLoB{:}@*)(*@\HLJLn{n}@*)
        (*@\HLJLn{ret}@*) (*@\HLJLoB{+=}@*) (*@\HLJLn{w}@*)(*@\HLJLp{[}@*)(*@\HLJLn{j}@*)(*@\HLJLp{]}@*)(*@\HLJLoB{*}@*)(*@\HLJLnf{exp}@*)(*@\HLJLp{(}@*)(*@\HLJLn{z}@*)(*@\HLJLp{[}@*)(*@\HLJLn{j}@*)(*@\HLJLp{])}@*)(*@\HLJLoB{*}@*)(*@\HLJLnf{inv}@*)(*@\HLJLp{(}@*)(*@\HLJLn{z}@*)(*@\HLJLp{[}@*)(*@\HLJLn{j}@*)(*@\HLJLp{]}@*)(*@\HLJLoB{*}@*)(*@\HLJLn{I}@*) (*@\HLJLoB{-}@*) (*@\HLJLn{A}@*)(*@\HLJLp{)}@*)
    (*@\HLJLk{end}@*)

    (*@\HLJLn{ret}@*)(*@\HLJLoB{/}@*)(*@\HLJLp{(}@*)(*@\HLJLni{2}@*)(*@\HLJLn{\ensuremath{\pi}}@*)(*@\HLJLoB{*}@*)(*@\HLJLn{im}@*)(*@\HLJLp{)}@*)
(*@\HLJLk{end}@*)
   
(*@\HLJLnf{circle{\_}exp}@*)(*@\HLJLp{(}@*)(*@\HLJLn{A}@*)(*@\HLJLp{,}@*) (*@\HLJLni{100}@*)(*@\HLJLp{,}@*) (*@\HLJLni{0}@*)(*@\HLJLp{,}@*) (*@\HLJLnfB{8.0}@*)(*@\HLJLp{)}@*) (*@\HLJLoB{-}@*)(*@\HLJLnf{exp}@*)(*@\HLJLp{(}@*)(*@\HLJLn{A}@*)(*@\HLJLp{)}@*) (*@\HLJLoB{|>}@*)(*@\HLJLn{norm}@*)
\end{lstlisting}

\begin{lstlisting}
Error: UndefVarError: periodic_rule not defined
\end{lstlisting}


In this case, it is beneficial to use an ellipse:


\begin{lstlisting}
(*@\HLJLk{function}@*) (*@\HLJLnf{ellipse{\_}exp}@*)(*@\HLJLp{(}@*)(*@\HLJLn{A}@*)(*@\HLJLp{,}@*) (*@\HLJLn{n}@*)(*@\HLJLp{,}@*) (*@\HLJLn{z\ensuremath{\_0}}@*)(*@\HLJLp{,}@*) (*@\HLJLn{a}@*)(*@\HLJLp{,}@*) (*@\HLJLn{b}@*)(*@\HLJLp{)}@*)
    (*@\HLJLn{z}@*)(*@\HLJLp{,}@*)(*@\HLJLn{w}@*) (*@\HLJLoB{=}@*) (*@\HLJLnf{ellipse{\_}rule}@*)(*@\HLJLp{(}@*)(*@\HLJLn{n}@*)(*@\HLJLp{,}@*)(*@\HLJLn{a}@*)(*@\HLJLp{,}@*)(*@\HLJLn{b}@*)(*@\HLJLp{)}@*)
    (*@\HLJLn{z}@*) (*@\HLJLoB{.+=}@*) (*@\HLJLn{z\ensuremath{\_0}}@*)

    (*@\HLJLn{ret}@*) (*@\HLJLoB{=}@*) (*@\HLJLnf{zero}@*)(*@\HLJLp{(}@*)(*@\HLJLn{A}@*)(*@\HLJLp{)}@*)
    (*@\HLJLk{for}@*) (*@\HLJLn{j}@*)(*@\HLJLoB{=}@*)(*@\HLJLni{1}@*)(*@\HLJLoB{:}@*)(*@\HLJLn{n}@*)
        (*@\HLJLn{ret}@*) (*@\HLJLoB{+=}@*) (*@\HLJLn{w}@*)(*@\HLJLp{[}@*)(*@\HLJLn{j}@*)(*@\HLJLp{]}@*)(*@\HLJLoB{*}@*)(*@\HLJLnf{exp}@*)(*@\HLJLp{(}@*)(*@\HLJLn{z}@*)(*@\HLJLp{[}@*)(*@\HLJLn{j}@*)(*@\HLJLp{])}@*)(*@\HLJLoB{*}@*)(*@\HLJLnf{inv}@*)(*@\HLJLp{(}@*)(*@\HLJLn{z}@*)(*@\HLJLp{[}@*)(*@\HLJLn{j}@*)(*@\HLJLp{]}@*)(*@\HLJLoB{*}@*)(*@\HLJLn{I}@*) (*@\HLJLoB{-}@*) (*@\HLJLn{A}@*)(*@\HLJLp{)}@*)
    (*@\HLJLk{end}@*)
    (*@\HLJLn{ret}@*)(*@\HLJLoB{/}@*)(*@\HLJLp{(}@*)(*@\HLJLni{2}@*)(*@\HLJLn{\ensuremath{\pi}}@*)(*@\HLJLoB{*}@*)(*@\HLJLn{im}@*)(*@\HLJLp{)}@*)
(*@\HLJLk{end}@*)


(*@\HLJLnf{ellipse{\_}exp}@*)(*@\HLJLp{(}@*)(*@\HLJLn{A}@*)(*@\HLJLp{,}@*) (*@\HLJLni{50}@*)(*@\HLJLp{,}@*) (*@\HLJLni{0}@*)(*@\HLJLp{,}@*) (*@\HLJLnfB{8.0}@*)(*@\HLJLp{,}@*) (*@\HLJLnfB{5.0}@*)(*@\HLJLp{)}@*) (*@\HLJLoB{-}@*)(*@\HLJLnf{exp}@*)(*@\HLJLp{(}@*)(*@\HLJLn{A}@*)(*@\HLJLp{)}@*) (*@\HLJLoB{|>}@*)(*@\HLJLn{norm}@*)
\end{lstlisting}

\begin{lstlisting}
Error: UndefVarError: ellipse_rule not defined
\end{lstlisting}


For matrices with large negative eigenvalues (For example, discretisations of the Laplacian), complex quadrature can lead to much better accuracy than Taylor series:


\begin{lstlisting}
(*@\HLJLk{function}@*) (*@\HLJLnf{taylor{\_}exp}@*)(*@\HLJLp{(}@*)(*@\HLJLn{A}@*)(*@\HLJLp{,}@*)(*@\HLJLn{n}@*)(*@\HLJLp{)}@*)
    (*@\HLJLn{ret}@*) (*@\HLJLoB{=}@*) (*@\HLJLnf{Matrix}@*)(*@\HLJLp{(}@*)(*@\HLJLn{I}@*)(*@\HLJLp{,}@*) (*@\HLJLnf{size}@*)(*@\HLJLp{(}@*)(*@\HLJLn{A}@*)(*@\HLJLp{))}@*)
    (*@\HLJLk{for}@*) (*@\HLJLn{k}@*)(*@\HLJLoB{=}@*)(*@\HLJLni{1}@*)(*@\HLJLoB{:}@*)(*@\HLJLn{n}@*)
        (*@\HLJLn{ret}@*) (*@\HLJLoB{+=}@*) (*@\HLJLn{A}@*)(*@\HLJLoB{{\textasciicircum}}@*)(*@\HLJLn{k}@*)(*@\HLJLoB{/}@*)(*@\HLJLnf{factorial}@*)(*@\HLJLp{(}@*)(*@\HLJLnfB{1.0}@*)(*@\HLJLn{k}@*)(*@\HLJLp{)}@*)
    (*@\HLJLk{end}@*)
    (*@\HLJLn{ret}@*)
(*@\HLJLk{end}@*)

(*@\HLJLn{B}@*) (*@\HLJLoB{=}@*) (*@\HLJLn{A}@*) (*@\HLJLoB{-}@*) (*@\HLJLni{20}@*)(*@\HLJLn{I}@*)

(*@\HLJLnf{taylor{\_}exp}@*)(*@\HLJLp{(}@*)(*@\HLJLn{B}@*)(*@\HLJLp{,}@*) (*@\HLJLni{200}@*)(*@\HLJLp{)}@*) (*@\HLJLoB{-}@*)(*@\HLJLnf{exp}@*)(*@\HLJLp{(}@*)(*@\HLJLn{B}@*)(*@\HLJLp{)}@*) (*@\HLJLoB{|>}@*)(*@\HLJLn{norm}@*)
\end{lstlisting}

\begin{lstlisting}
2.2269613035365295e-7
\end{lstlisting}


We can use an ellpise to surround the spectrum:


\begin{lstlisting}
(*@\HLJLnf{scatter}@*)(*@\HLJLp{(}@*)(*@\HLJLn{complex}@*)(*@\HLJLoB{.}@*)(*@\HLJLp{(}@*)(*@\HLJLnf{eigvals}@*)(*@\HLJLp{(}@*)(*@\HLJLn{B}@*)(*@\HLJLp{)))}@*)
\end{lstlisting}

\begin{lstlisting}
Error: UndefVarError: scatter not defined
\end{lstlisting}


\begin{lstlisting}
(*@\HLJLnf{plot!}@*)(*@\HLJLp{(}@*)(*@\HLJLnf{ellipse{\_}rule}@*)(*@\HLJLp{(}@*)(*@\HLJLni{50}@*)(*@\HLJLp{,}@*)(*@\HLJLni{8}@*)(*@\HLJLp{,}@*)(*@\HLJLni{5}@*)(*@\HLJLp{)[}@*)(*@\HLJLni{1}@*)(*@\HLJLp{]}@*) (*@\HLJLoB{.-}@*) (*@\HLJLni{20}@*)(*@\HLJLp{)}@*)
\end{lstlisting}

\begin{lstlisting}
Error: UndefVarError: ellipse_rule not defined
\end{lstlisting}


\begin{lstlisting}
(*@\HLJLnf{norm}@*)(*@\HLJLp{(}@*)(*@\HLJLnf{ellipse{\_}exp}@*)(*@\HLJLp{(}@*)(*@\HLJLn{B}@*)(*@\HLJLp{,}@*) (*@\HLJLni{50}@*)(*@\HLJLp{,}@*) (*@\HLJLoB{-}@*)(*@\HLJLnfB{20.0}@*)(*@\HLJLp{,}@*) (*@\HLJLnfB{8.0}@*)(*@\HLJLp{,}@*) (*@\HLJLnfB{5.0}@*)(*@\HLJLp{)}@*) (*@\HLJLoB{-}@*) (*@\HLJLnf{exp}@*)(*@\HLJLp{(}@*)(*@\HLJLn{B}@*)(*@\HLJLp{))}@*)
\end{lstlisting}

\begin{lstlisting}
Error: UndefVarError: ellipse_rule not defined
\end{lstlisting}


\subsubsection{Gershgorin circle theorem}
If we only know $A$, how do we know how big to make the contour? Gershgorin's circle theorem gives the answer:

\textbf{Theorem (Gershgorin)} Let $A \in {\mathbb C}^{d \times d}$ and define 

\[
R_k = \sum_{j=1 \atop j \neq k}^d |a_{kj}| 
\]
Then 

\[
\rho(A) \subset \bigcup_{k=1}^d \bar B(a_{kk}, R_k)
\]
where $\bar B(z_0, r)$ is the closed disk of radius $r$ and $\rho(A)$ is the set of eigenvalues.

**Proof **

\ensuremath{\blacksquare}

\emph{Demonstration} Here we apply this to a particular matrix:


\begin{lstlisting}
(*@\HLJLn{A}@*) (*@\HLJLoB{=}@*) (*@\HLJLp{[}@*)(*@\HLJLni{1}@*) (*@\HLJLni{2}@*) (*@\HLJLni{3}@*)(*@\HLJLp{;}@*) (*@\HLJLni{1}@*) (*@\HLJLni{5}@*) (*@\HLJLni{2}@*)(*@\HLJLp{;}@*) (*@\HLJLoB{-}@*)(*@\HLJLni{4}@*) (*@\HLJLni{1}@*) (*@\HLJLni{6}@*)(*@\HLJLp{]}@*)
\end{lstlisting}

\begin{lstlisting}
3(*@\ensuremath{\times}@*)3 Array{Int64,2}:
  1  2  3
  1  5  2
 -4  1  6
\end{lstlisting}


The following calculates the row sums:


\begin{lstlisting}
(*@\HLJLn{R}@*) (*@\HLJLoB{=}@*) (*@\HLJLnf{sum}@*)(*@\HLJLp{(}@*)(*@\HLJLn{abs}@*)(*@\HLJLoB{.}@*)(*@\HLJLp{(}@*)(*@\HLJLn{A}@*) (*@\HLJLoB{-}@*) (*@\HLJLnf{Diagonal}@*)(*@\HLJLp{(}@*)(*@\HLJLnf{diag}@*)(*@\HLJLp{(}@*)(*@\HLJLn{A}@*)(*@\HLJLp{))),}@*)(*@\HLJLn{dims}@*)(*@\HLJLoB{=}@*)(*@\HLJLni{2}@*)(*@\HLJLp{)}@*)
\end{lstlisting}

\begin{lstlisting}
3(*@\ensuremath{\times}@*)1 Array{Int64,2}:
 5
 3
 5
\end{lstlisting}


Gershgorin's theorem tells us that the spectrum lies in the union of the circles surrounding the diagonals:


\begin{lstlisting}
(*@\HLJLnf{drawcircle!}@*)(*@\HLJLp{(}@*)(*@\HLJLn{z0}@*)(*@\HLJLp{,}@*) (*@\HLJLn{R}@*)(*@\HLJLp{)}@*) (*@\HLJLoB{=}@*) (*@\HLJLnf{plot!}@*)(*@\HLJLp{(}@*)(*@\HLJLn{\ensuremath{\theta}}@*)(*@\HLJLoB{->}@*) (*@\HLJLnf{real}@*)(*@\HLJLp{(}@*)(*@\HLJLn{z0}@*)(*@\HLJLp{)}@*) (*@\HLJLoB{+}@*) (*@\HLJLn{R}@*)(*@\HLJLp{[}@*)(*@\HLJLni{1}@*)(*@\HLJLp{]}@*)(*@\HLJLoB{*}@*)(*@\HLJLnf{cos}@*)(*@\HLJLp{(}@*)(*@\HLJLn{\ensuremath{\theta}}@*)(*@\HLJLp{),}@*) (*@\HLJLn{\ensuremath{\theta}}@*)(*@\HLJLoB{->}@*) (*@\HLJLnf{imag}@*)(*@\HLJLp{(}@*)(*@\HLJLn{z0}@*)(*@\HLJLp{)}@*) (*@\HLJLoB{+}@*) (*@\HLJLn{R}@*)(*@\HLJLp{[}@*)(*@\HLJLni{1}@*)(*@\HLJLp{]}@*)(*@\HLJLoB{*}@*)(*@\HLJLnf{sin}@*)(*@\HLJLp{(}@*)(*@\HLJLn{\ensuremath{\theta}}@*)(*@\HLJLp{),}@*) (*@\HLJLni{0}@*)(*@\HLJLp{,}@*) (*@\HLJLni{2}@*)(*@\HLJLn{\ensuremath{\pi}}@*)(*@\HLJLp{,}@*) (*@\HLJLn{fill}@*)(*@\HLJLoB{=}@*)(*@\HLJLp{(}@*)(*@\HLJLni{0}@*)(*@\HLJLp{,}@*)(*@\HLJLoB{:}@*)(*@\HLJLn{red}@*)(*@\HLJLp{),}@*) (*@\HLJLn{\ensuremath{\alpha}}@*) (*@\HLJLoB{=}@*) (*@\HLJLnfB{0.2}@*)(*@\HLJLp{,}@*) (*@\HLJLn{legend}@*)(*@\HLJLoB{=}@*)(*@\HLJLkc{false}@*)(*@\HLJLp{)}@*)

(*@\HLJLn{\ensuremath{\lambda}}@*) (*@\HLJLoB{=}@*) (*@\HLJLnf{eigvals}@*)(*@\HLJLp{(}@*)(*@\HLJLn{A}@*)(*@\HLJLp{)}@*)
(*@\HLJLn{p}@*) (*@\HLJLoB{=}@*) (*@\HLJLnf{plot}@*)(*@\HLJLp{()}@*)
\end{lstlisting}

\begin{lstlisting}
Error: UndefVarError: plot not defined
\end{lstlisting}


\begin{lstlisting}
(*@\HLJLk{for}@*) (*@\HLJLn{k}@*) (*@\HLJLoB{=}@*) (*@\HLJLni{1}@*)(*@\HLJLoB{:}@*)(*@\HLJLnf{size}@*)(*@\HLJLp{(}@*)(*@\HLJLn{A}@*)(*@\HLJLp{,}@*)(*@\HLJLni{1}@*)(*@\HLJLp{)}@*)
    (*@\HLJLnf{drawcircle!}@*)(*@\HLJLp{(}@*)(*@\HLJLn{A}@*)(*@\HLJLp{[}@*)(*@\HLJLn{k}@*)(*@\HLJLp{,}@*)(*@\HLJLn{k}@*)(*@\HLJLp{],}@*) (*@\HLJLn{R}@*)(*@\HLJLp{[}@*)(*@\HLJLn{k}@*)(*@\HLJLp{])}@*)
(*@\HLJLk{end}@*)
\end{lstlisting}

\begin{lstlisting}
Error: UndefVarError: plot! not defined
\end{lstlisting}


\begin{lstlisting}
(*@\HLJLnf{scatter!}@*)(*@\HLJLp{(}@*)(*@\HLJLn{complex}@*)(*@\HLJLoB{.}@*)(*@\HLJLp{(}@*)(*@\HLJLn{\ensuremath{\lambda}}@*)(*@\HLJLp{);}@*) (*@\HLJLn{label}@*)(*@\HLJLoB{=}@*)(*@\HLJLs{"{}eigenvalues"{}}@*)(*@\HLJLp{)}@*)
\end{lstlisting}

\begin{lstlisting}
Error: UndefVarError: scatter! not defined
\end{lstlisting}


\begin{lstlisting}
(*@\HLJLnf{scatter!}@*)(*@\HLJLp{(}@*)(*@\HLJLn{complex}@*)(*@\HLJLoB{.}@*)(*@\HLJLp{(}@*)(*@\HLJLnf{diag}@*)(*@\HLJLp{(}@*)(*@\HLJLn{A}@*)(*@\HLJLp{));}@*) (*@\HLJLn{label}@*)(*@\HLJLoB{=}@*)(*@\HLJLs{"{}diagonals"{}}@*)(*@\HLJLp{)}@*)
\end{lstlisting}

\begin{lstlisting}
Error: UndefVarError: scatter! not defined
\end{lstlisting}


\begin{lstlisting}
(*@\HLJLn{p}@*)
\end{lstlisting}

\begin{lstlisting}
Error: UndefVarError: p not defined
\end{lstlisting}


We can therefore use this to choose a contour big enough to surround all the circles. Here's a fairly simplistic construction for our case where everything is real:


\begin{lstlisting}
(*@\HLJLn{z\ensuremath{\_0}}@*) (*@\HLJLoB{=}@*) (*@\HLJLp{(}@*)(*@\HLJLnf{maximum}@*)(*@\HLJLp{(}@*)(*@\HLJLnf{diag}@*)(*@\HLJLp{(}@*)(*@\HLJLn{A}@*)(*@\HLJLp{)}@*) (*@\HLJLoB{.+}@*) (*@\HLJLn{R}@*)(*@\HLJLp{)}@*) (*@\HLJLoB{+}@*) (*@\HLJLnf{minimum}@*)(*@\HLJLp{(}@*)(*@\HLJLnf{diag}@*)(*@\HLJLp{(}@*)(*@\HLJLn{A}@*)(*@\HLJLp{)}@*) (*@\HLJLoB{.-}@*) (*@\HLJLn{R}@*)(*@\HLJLp{))}@*) (*@\HLJLoB{/}@*)(*@\HLJLni{2}@*) (*@\HLJLcs{{\#}}@*) (*@\HLJLcs{average}@*) (*@\HLJLcs{edges}@*) (*@\HLJLcs{of}@*) (*@\HLJLcs{circle}@*)
(*@\HLJLn{r}@*) (*@\HLJLoB{=}@*) (*@\HLJLnf{max}@*)(*@\HLJLp{(}@*)(*@\HLJLn{abs}@*)(*@\HLJLoB{.}@*)(*@\HLJLp{(}@*)(*@\HLJLnf{diag}@*)(*@\HLJLp{(}@*)(*@\HLJLn{A}@*)(*@\HLJLp{)}@*) (*@\HLJLoB{.-}@*) (*@\HLJLn{R}@*) (*@\HLJLoB{.-}@*) (*@\HLJLn{z\ensuremath{\_0}}@*)(*@\HLJLp{)}@*)(*@\HLJLoB{...}@*)(*@\HLJLp{,}@*) (*@\HLJLn{abs}@*)(*@\HLJLoB{.}@*)(*@\HLJLp{(}@*)(*@\HLJLnf{diag}@*)(*@\HLJLp{(}@*)(*@\HLJLn{A}@*)(*@\HLJLp{)}@*) (*@\HLJLoB{.+}@*) (*@\HLJLn{R}@*) (*@\HLJLoB{.-}@*) (*@\HLJLn{z\ensuremath{\_0}}@*)(*@\HLJLp{)}@*)(*@\HLJLoB{...}@*)(*@\HLJLp{)}@*)

(*@\HLJLnf{plot!}@*)(*@\HLJLp{(}@*)(*@\HLJLnf{Circle}@*)(*@\HLJLp{(}@*)(*@\HLJLn{z\ensuremath{\_0}}@*)(*@\HLJLp{,}@*) (*@\HLJLn{r}@*)(*@\HLJLp{))}@*)
\end{lstlisting}

\begin{lstlisting}
Error: UndefVarError: Circle not defined
\end{lstlisting}


Here we consider two cases


\begin{lstlisting}
(*@\HLJLn{N}@*) (*@\HLJLoB{=}@*) (*@\HLJLni{1000}@*)
(*@\HLJLn{h}@*) (*@\HLJLoB{=}@*) (*@\HLJLni{1}@*)(*@\HLJLoB{/}@*)(*@\HLJLn{N}@*)
(*@\HLJLn{\ensuremath{\Delta}}@*) (*@\HLJLoB{=}@*) (*@\HLJLnf{SymTridiagonal}@*)(*@\HLJLp{(}@*)(*@\HLJLnf{fill}@*)(*@\HLJLp{(}@*)(*@\HLJLoB{-}@*)(*@\HLJLni{2}@*)(*@\HLJLp{,}@*)(*@\HLJLn{N}@*)(*@\HLJLp{),}@*) (*@\HLJLnf{fill}@*)(*@\HLJLp{(}@*)(*@\HLJLni{1}@*)(*@\HLJLp{,}@*)(*@\HLJLn{N}@*)(*@\HLJLoB{-}@*)(*@\HLJLni{1}@*)(*@\HLJLp{))}@*)(*@\HLJLoB{/}@*)(*@\HLJLn{h}@*)(*@\HLJLoB{{\textasciicircum}}@*)(*@\HLJLni{2}@*)
(*@\HLJLnf{eigvals}@*)(*@\HLJLp{(}@*)(*@\HLJLn{\ensuremath{\Delta}}@*)(*@\HLJLp{)}@*)


(*@\HLJLn{u0}@*) (*@\HLJLoB{=}@*) (*@\HLJLn{x}@*) (*@\HLJLoB{->}@*) (*@\HLJLn{x}@*) (*@\HLJLoB{*}@*) (*@\HLJLp{(}@*)(*@\HLJLni{1}@*)(*@\HLJLoB{-}@*)(*@\HLJLn{x}@*)(*@\HLJLp{)}@*) (*@\HLJLoB{*}@*) (*@\HLJLnf{exp}@*)(*@\HLJLp{(}@*)(*@\HLJLn{x}@*)(*@\HLJLp{)}@*)

(*@\HLJLn{t}@*) (*@\HLJLoB{=}@*) (*@\HLJLnfB{0.1}@*)
(*@\HLJLn{xx}@*) (*@\HLJLoB{=}@*) (*@\HLJLnf{range}@*)(*@\HLJLp{(}@*)(*@\HLJLni{0}@*)(*@\HLJLp{,}@*)(*@\HLJLni{1}@*)(*@\HLJLp{;}@*) (*@\HLJLn{length}@*)(*@\HLJLoB{=}@*)(*@\HLJLn{N}@*)(*@\HLJLoB{+}@*)(*@\HLJLni{2}@*)(*@\HLJLp{)}@*)
(*@\HLJLn{A}@*) (*@\HLJLoB{=}@*) (*@\HLJLnf{Matrix}@*)(*@\HLJLp{(}@*)(*@\HLJLn{\ensuremath{\Delta}}@*)(*@\HLJLp{)}@*) (*@\HLJLcs{{\#}}@*) (*@\HLJLcs{Need}@*) (*@\HLJLcs{to}@*) (*@\HLJLcs{convert}@*) (*@\HLJLcs{to}@*) (*@\HLJLcs{dense}@*) (*@\HLJLcs{matrix}@*) (*@\HLJLcs{to}@*) (*@\HLJLcs{diagonalise}@*)
(*@\HLJLnf{plot}@*)(*@\HLJLp{(}@*)(*@\HLJLn{xx}@*)(*@\HLJLp{,}@*) (*@\HLJLp{[}@*)(*@\HLJLni{0}@*)(*@\HLJLp{;}@*) (*@\HLJLnf{exp}@*)(*@\HLJLp{(}@*)(*@\HLJLn{A}@*)(*@\HLJLoB{*}@*)(*@\HLJLn{t}@*)(*@\HLJLp{)}@*)(*@\HLJLoB{*}@*)(*@\HLJLn{u0}@*)(*@\HLJLoB{.}@*)(*@\HLJLp{(}@*)(*@\HLJLn{xx}@*)(*@\HLJLp{[}@*)(*@\HLJLni{2}@*)(*@\HLJLoB{:}@*)(*@\HLJLk{end}@*)(*@\HLJLoB{-}@*)(*@\HLJLni{1}@*)(*@\HLJLp{]);}@*) (*@\HLJLni{0}@*)(*@\HLJLp{])}@*)
\end{lstlisting}

\begin{lstlisting}
Error: UndefVarError: plot not defined
\end{lstlisting}


\begin{lstlisting}
(*@\HLJLnf{plot!}@*)(*@\HLJLp{(}@*)(*@\HLJLn{xx}@*)(*@\HLJLp{,}@*) (*@\HLJLp{[}@*)(*@\HLJLni{0}@*)(*@\HLJLp{;}@*) (*@\HLJLnf{exp}@*)(*@\HLJLp{(}@*)(*@\HLJLoB{-}@*)(*@\HLJLnf{sqrt}@*)(*@\HLJLp{(}@*)(*@\HLJLoB{-}@*)(*@\HLJLn{A}@*)(*@\HLJLp{)}@*)(*@\HLJLoB{*}@*)(*@\HLJLn{t}@*)(*@\HLJLp{)}@*)(*@\HLJLoB{*}@*)(*@\HLJLn{u0}@*)(*@\HLJLoB{.}@*)(*@\HLJLp{(}@*)(*@\HLJLn{xx}@*)(*@\HLJLp{[}@*)(*@\HLJLni{2}@*)(*@\HLJLoB{:}@*)(*@\HLJLk{end}@*)(*@\HLJLoB{-}@*)(*@\HLJLni{1}@*)(*@\HLJLp{]);}@*) (*@\HLJLni{0}@*)(*@\HLJLp{])}@*)
\end{lstlisting}

\begin{lstlisting}
Error: UndefVarError: plot! not defined
\end{lstlisting}


Slow!


\begin{lstlisting}
(*@\HLJLn{t}@*)(*@\HLJLoB{=}@*)(*@\HLJLni{10}@*)
(*@\HLJLk{using}@*) (*@\HLJLn{BenchmarkTools}@*)
(*@\HLJLnd{@btime}@*) (*@\HLJLnf{exp}@*)(*@\HLJLp{(}@*)(*@\HLJLn{A}@*)(*@\HLJLoB{*}@*)(*@\HLJLn{t}@*)(*@\HLJLp{)}@*)
\end{lstlisting}

\begin{lstlisting}
178.308 ms (28 allocations: 53.77 MiB)
1000(*@\ensuremath{\times}@*)1000 Array{Float64,2}:
 3.28482e-51  6.56961e-51  9.85434e-51  (*@\ensuremath{\dots}@*)  6.56961e-51  3.28482e-51
 6.56961e-51  1.31392e-50  1.97086e-50     1.31392e-50  6.56961e-51
 9.85434e-51  1.97086e-50  2.95626e-50     1.97086e-50  9.85434e-51
 1.3139e-50   2.62778e-50  3.94164e-50     2.62778e-50  1.3139e-50 
 1.64235e-50  3.28468e-50  4.92697e-50     3.28468e-50  1.64235e-50
 1.97078e-50  3.94154e-50  5.91226e-50  (*@\ensuremath{\dots}@*)  3.94154e-50  1.97078e-50
 2.29919e-50  4.59837e-50  6.89749e-50     4.59837e-50  2.29919e-50
 2.62759e-50  5.25514e-50  7.88265e-50     5.25514e-50  2.62759e-50
 2.95595e-50  5.91187e-50  8.86774e-50     5.91187e-50  2.95595e-50
 3.28429e-50  6.56854e-50  9.85273e-50     6.56854e-50  3.28429e-50
 (*@\ensuremath{\vdots}@*)                                      (*@\ensuremath{\ddots}@*)                          
 2.95595e-50  5.91187e-50  8.86774e-50     5.91187e-50  2.95595e-50
 2.62759e-50  5.25514e-50  7.88265e-50     5.25514e-50  2.62759e-50
 2.29919e-50  4.59837e-50  6.89749e-50     4.59837e-50  2.29919e-50
 1.97078e-50  3.94154e-50  5.91226e-50     3.94154e-50  1.97078e-50
 1.64235e-50  3.28468e-50  4.92697e-50  (*@\ensuremath{\dots}@*)  3.28468e-50  1.64235e-50
 1.3139e-50   2.62778e-50  3.94164e-50     2.62778e-50  1.3139e-50 
 9.85434e-51  1.97086e-50  2.95626e-50     1.97086e-50  9.85434e-51
 6.56961e-51  1.31392e-50  1.97086e-50     1.31392e-50  6.56961e-51
 3.28482e-51  6.56961e-51  9.85434e-51     6.56961e-51  3.28482e-51
\end{lstlisting}


Inverse is fast!


\begin{lstlisting}
(*@\HLJLn{v0}@*) (*@\HLJLoB{=}@*) (*@\HLJLn{u0}@*)(*@\HLJLoB{.}@*)(*@\HLJLp{(}@*)(*@\HLJLn{xx}@*)(*@\HLJLp{[}@*)(*@\HLJLni{2}@*)(*@\HLJLoB{:}@*)(*@\HLJLk{end}@*)(*@\HLJLoB{-}@*)(*@\HLJLni{1}@*)(*@\HLJLp{])}@*)
(*@\HLJLnd{@time}@*) (*@\HLJLp{(}@*)(*@\HLJLn{\ensuremath{\Delta}}@*)(*@\HLJLoB{-}@*)(*@\HLJLnfB{0.1}@*)(*@\HLJLn{im}@*)(*@\HLJLoB{*}@*)(*@\HLJLn{I}@*)(*@\HLJLp{)}@*) (*@\HLJLoB{{\textbackslash}}@*) (*@\HLJLn{v0}@*)
\end{lstlisting}

\begin{lstlisting}
0.000054 seconds (15 allocations: 79.094 KiB)
1000-element Array{Complex{Float64},1}:
 -0.00012698534640807264 + 1.3694267632523385e-6im
 -0.00025396969395258777 + 2.7388408279700356e-6im
  -0.0003809520437729823 + 4.108229495718337e-6im 
  -0.0005079313970146893 + 5.4775800682622626e-6im
  -0.0006349067548321476 + 6.846879847666487e-6im 
  -0.0007618771183918197 + 8.216116136395228e-6im 
  -0.0008888414888752172 + 9.58527623741213e-6im  
  -0.0010157988674819353 + 1.0954347454280142e-5im
  -0.0011427482554326937 + 1.2323317091261413e-5im
   -0.001269688653972387 + 1.3692172453417137e-5im
                         (*@\ensuremath{\vdots}@*)                        
  -0.0013945493469472022 + 1.2921824147992476e-5im
  -0.0012396601917373797 + 1.14864171677115e-5im  
  -0.0010847496583173222 + 1.0050886221411347e-5im
  -0.0009298203814081073 + 8.615246800145361e-6im 
  -0.0007748750064125377 + 7.179514396841236e-6im 
  -0.0006199161894392533 + 5.743704506036466e-6im 
  -0.0004649465973268873 + 4.307832623612753e-6im 
 -0.00030996890766826464 + 2.8719142465293073e-6im
 -0.00015498580883464345 + 1.4359648725550954e-6im
\end{lstlisting}



\end{document}
