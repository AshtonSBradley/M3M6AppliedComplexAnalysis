\documentclass[12pt,a4paper]{article}

\usepackage[a4paper,text={16.5cm,25.2cm},centering]{geometry}
\usepackage{lmodern}
\usepackage{amssymb,amsmath}
\usepackage{bm}
\usepackage{graphicx}
\usepackage{microtype}
\usepackage{hyperref}
\setlength{\parindent}{0pt}
\setlength{\parskip}{1.2ex}

\hypersetup
       {   pdfauthor = { Sheehan Olver },
           pdftitle={ foo },
           colorlinks=TRUE,
           linkcolor=black,
           citecolor=blue,
           urlcolor=blue
       }




\usepackage{upquote}
\usepackage{listings}
\usepackage{xcolor}
\lstset{
    basicstyle=\ttfamily\footnotesize,
    upquote=true,
    breaklines=true,
    breakindent=0pt,
    keepspaces=true,
    showspaces=false,
    columns=fullflexible,
    showtabs=false,
    showstringspaces=false,
    escapeinside={(*@}{@*)},
    extendedchars=true,
}
\newcommand{\HLJLt}[1]{#1}
\newcommand{\HLJLw}[1]{#1}
\newcommand{\HLJLe}[1]{#1}
\newcommand{\HLJLeB}[1]{#1}
\newcommand{\HLJLo}[1]{#1}
\newcommand{\HLJLk}[1]{\textcolor[RGB]{148,91,176}{\textbf{#1}}}
\newcommand{\HLJLkc}[1]{\textcolor[RGB]{59,151,46}{\textit{#1}}}
\newcommand{\HLJLkd}[1]{\textcolor[RGB]{214,102,97}{\textit{#1}}}
\newcommand{\HLJLkn}[1]{\textcolor[RGB]{148,91,176}{\textbf{#1}}}
\newcommand{\HLJLkp}[1]{\textcolor[RGB]{148,91,176}{\textbf{#1}}}
\newcommand{\HLJLkr}[1]{\textcolor[RGB]{148,91,176}{\textbf{#1}}}
\newcommand{\HLJLkt}[1]{\textcolor[RGB]{148,91,176}{\textbf{#1}}}
\newcommand{\HLJLn}[1]{#1}
\newcommand{\HLJLna}[1]{#1}
\newcommand{\HLJLnb}[1]{#1}
\newcommand{\HLJLnbp}[1]{#1}
\newcommand{\HLJLnc}[1]{#1}
\newcommand{\HLJLncB}[1]{#1}
\newcommand{\HLJLnd}[1]{\textcolor[RGB]{214,102,97}{#1}}
\newcommand{\HLJLne}[1]{#1}
\newcommand{\HLJLneB}[1]{#1}
\newcommand{\HLJLnf}[1]{\textcolor[RGB]{66,102,213}{#1}}
\newcommand{\HLJLnfm}[1]{\textcolor[RGB]{66,102,213}{#1}}
\newcommand{\HLJLnp}[1]{#1}
\newcommand{\HLJLnl}[1]{#1}
\newcommand{\HLJLnn}[1]{#1}
\newcommand{\HLJLno}[1]{#1}
\newcommand{\HLJLnt}[1]{#1}
\newcommand{\HLJLnv}[1]{#1}
\newcommand{\HLJLnvc}[1]{#1}
\newcommand{\HLJLnvg}[1]{#1}
\newcommand{\HLJLnvi}[1]{#1}
\newcommand{\HLJLnvm}[1]{#1}
\newcommand{\HLJLl}[1]{#1}
\newcommand{\HLJLld}[1]{\textcolor[RGB]{148,91,176}{\textit{#1}}}
\newcommand{\HLJLs}[1]{\textcolor[RGB]{201,61,57}{#1}}
\newcommand{\HLJLsa}[1]{\textcolor[RGB]{201,61,57}{#1}}
\newcommand{\HLJLsb}[1]{\textcolor[RGB]{201,61,57}{#1}}
\newcommand{\HLJLsc}[1]{\textcolor[RGB]{201,61,57}{#1}}
\newcommand{\HLJLsd}[1]{\textcolor[RGB]{201,61,57}{#1}}
\newcommand{\HLJLsdB}[1]{\textcolor[RGB]{201,61,57}{#1}}
\newcommand{\HLJLsdC}[1]{\textcolor[RGB]{201,61,57}{#1}}
\newcommand{\HLJLse}[1]{\textcolor[RGB]{59,151,46}{#1}}
\newcommand{\HLJLsh}[1]{\textcolor[RGB]{201,61,57}{#1}}
\newcommand{\HLJLsi}[1]{#1}
\newcommand{\HLJLso}[1]{\textcolor[RGB]{201,61,57}{#1}}
\newcommand{\HLJLsr}[1]{\textcolor[RGB]{201,61,57}{#1}}
\newcommand{\HLJLss}[1]{\textcolor[RGB]{201,61,57}{#1}}
\newcommand{\HLJLssB}[1]{\textcolor[RGB]{201,61,57}{#1}}
\newcommand{\HLJLnB}[1]{\textcolor[RGB]{59,151,46}{#1}}
\newcommand{\HLJLnbB}[1]{\textcolor[RGB]{59,151,46}{#1}}
\newcommand{\HLJLnfB}[1]{\textcolor[RGB]{59,151,46}{#1}}
\newcommand{\HLJLnh}[1]{\textcolor[RGB]{59,151,46}{#1}}
\newcommand{\HLJLni}[1]{\textcolor[RGB]{59,151,46}{#1}}
\newcommand{\HLJLnil}[1]{\textcolor[RGB]{59,151,46}{#1}}
\newcommand{\HLJLnoB}[1]{\textcolor[RGB]{59,151,46}{#1}}
\newcommand{\HLJLoB}[1]{\textcolor[RGB]{102,102,102}{\textbf{#1}}}
\newcommand{\HLJLow}[1]{\textcolor[RGB]{102,102,102}{\textbf{#1}}}
\newcommand{\HLJLp}[1]{#1}
\newcommand{\HLJLc}[1]{\textcolor[RGB]{153,153,119}{\textit{#1}}}
\newcommand{\HLJLch}[1]{\textcolor[RGB]{153,153,119}{\textit{#1}}}
\newcommand{\HLJLcm}[1]{\textcolor[RGB]{153,153,119}{\textit{#1}}}
\newcommand{\HLJLcp}[1]{\textcolor[RGB]{153,153,119}{\textit{#1}}}
\newcommand{\HLJLcpB}[1]{\textcolor[RGB]{153,153,119}{\textit{#1}}}
\newcommand{\HLJLcs}[1]{\textcolor[RGB]{153,153,119}{\textit{#1}}}
\newcommand{\HLJLcsB}[1]{\textcolor[RGB]{153,153,119}{\textit{#1}}}
\newcommand{\HLJLg}[1]{#1}
\newcommand{\HLJLgd}[1]{#1}
\newcommand{\HLJLge}[1]{#1}
\newcommand{\HLJLgeB}[1]{#1}
\newcommand{\HLJLgh}[1]{#1}
\newcommand{\HLJLgi}[1]{#1}
\newcommand{\HLJLgo}[1]{#1}
\newcommand{\HLJLgp}[1]{#1}
\newcommand{\HLJLgs}[1]{#1}
\newcommand{\HLJLgsB}[1]{#1}
\newcommand{\HLJLgt}[1]{#1}



\def\qqand{\qquad\hbox{and}\qquad}
\def\qqfor{\qquad\hbox{for}\qquad}
\def\qqas{\qquad\hbox{as}\qquad}
\def\half{ {1 \over 2} }
\def\D{ {\rm d} }
\def\I{ {\rm i} }
\def\E{ {\rm e} }
\def\C{ {\mathbb C} }
\def\R{ {\mathbb R} }
\def\Z{ {\mathbb Z} }
\def\CC{ {\cal C} }
\def\HH{ {\cal H} }
\def\LL{ {\cal L} }
\def\vc#1{ {\mathbf #1} }
\def\bbC{ {\mathbb C} }

\def\qqqquad{\qquad\qquad}
\def\qqwhere{\qquad\hbox{where}\qquad}
\def\Res_#1{\underset{#1}{\rm Res}\,}
\def\sech{ {\rm sech}\, }
\def\acos{ {\rm acos}\, }
\def\asin{ {\rm asin}\, }
\def\atan{ {\rm atan}\, }
\def\Ei{ {\rm Ei}\, }
\def\upepsilon{\varepsilon}


\def\Xint#1{ \mathchoice
   {\XXint\displaystyle\textstyle{#1} }%
   {\XXint\textstyle\scriptstyle{#1} }%
   {\XXint\scriptstyle\scriptscriptstyle{#1} }%
   {\XXint\scriptscriptstyle\scriptscriptstyle{#1} }%
   \!\int}
\def\XXint#1#2#3{ {\setbox0=\hbox{$#1{#2#3}{\int}$}
     \vcenter{\hbox{$#2#3$}}\kern-.5\wd0} }
\def\ddashint{\Xint=}
\def\dashint{\Xint-}
% \def\dashint
\def\infdashint{\dashint_{-\infty}^\infty}




\def\addtab#1={#1\;&=}
\def\ccr{\\\addtab}
\def\ip<#1>{\left\langle{#1}\right\rangle}
\def\dx{\D x}
\def\dt{\D t}
\def\dz{\D z}

\def\norm#1{\left\| #1 \right\|}

\def\pr(#1){\left({#1}\right)}
\def\br[#1]{\left[{#1}\right]}

\def\abs#1{\left|{#1}\right|}
\def\fpr(#1){\!\pr({#1})}

\def\sopmatrix#1{ \begin{pmatrix}#1\end{pmatrix} }

\def\endash{–}
\def\mdblksquare{\blacksquare}
\def\lgblksquare{\blacksquare}
\def\scre{\E}
\def\mapengine#1,#2.{\mapfunction{#1}\ifx\void#2\else\mapengine #2.\fi }

\def\map[#1]{\mapengine #1,\void.}

\def\mapenginesep_#1#2,#3.{\mapfunction{#2}\ifx\void#3\else#1\mapengine #3.\fi }

\def\mapsep_#1[#2]{\mapenginesep_{#1}#2,\void.}


\def\vcbr[#1]{\pr(#1)}


\def\bvect[#1,#2]{
{
\def\dots{\cdots}
\def\mapfunction##1{\ | \  ##1}
	\sopmatrix{
		 \,#1\map[#2]\,
	}
}
}



\def\vect[#1]{
{\def\dots{\ldots}
	\vcbr[{#1}]
} }

\def\vectt[#1]{
{\def\dots{\ldots}
	\vect[{#1}]^{\top}
} }

\def\Vectt[#1]{
{
\def\mapfunction##1{##1 \cr} 
\def\dots{\vdots}
	\begin{pmatrix}
		\map[#1]
	\end{pmatrix}
} }

\def\addtab#1={#1\;&=}
\def\ccr{\\\addtab}

\def\questionequals{= \!\!\!\!\!\!{\scriptstyle ? \atop }\,\,\,}

\begin{document}

\textbf{M3M6: Applied Complex Analysis}

Dr. Sheehan Olver

s.olver@imperial.ac.uk

\section{Lecture 26: Laurent and Toeplitz operators}
This lecture concerns the computation of the UL factorisation of Toeplitz operators, which are infinite matrices with constant diagonals.  It turns out this can be reduced to solving an RH problem on the unit circle.  We first discuss the case of inverting Laurent operators, which are bi-infinite matrices with constant diagonals.

\subsection{Laurent operators}
A \emph{Laurent operator} is a bi-infinite matrix with constant diagonals. We associate it with a Laurent series on the unit circle called the \emph{symbol}:

\[
a(z) = \sum_{k=-\infty}^\infty a_k z^k
\]
Then the corresponding Laurent operator is

\[
L[a(z)] := \sopmatrix{
\ddots & \ddots & \ddots& \ddots& \ddots& \ddots& \ddots \\
\ddots  & a_1 & a_0 & a_{-1} & a_{-2} & a_{-3} & \ddots \\
\ddots  & a_2 & a_1 & a_0 & a_{-1} & a_{-2} & \ddots \\
\ddots  & a_3 & a_2 & a_1 & a_0 & a_{-1} &  \ddots \\
\ddots & \ddots & \ddots& \ddots& \ddots& \ddots& \ddots
}
\]
In other words, for $k,j \in \Z$ we have

\[
\vc e_k^\top L[a(z)] \vc e_j = a_{k-j}
\]
Note that $L[a(z)] \vc b$ is a discrete convolution:

\[
\vc e_k^\top L[a(z)] \vc b = \sum_{j=-\infty}^\infty a_{k-j} b_j
\]
and gives the Laurent coefficients of $a(z) b(z)$ where

\[
b(z) = \sum_{k=-\infty}^\infty b_k z^k
\]
This implies an algebraic structure: we have 

\[
L[a(z)] L[b(z)] = L[a(z)b(z)] = L[b(z)] L[a(z)]
\]
in particular, provided $a(z) \neq 0$ on the unit circle,

\[
L[a(z)]^{-1} = L[a(z)^{-1}]
\]
\subsubsection{Example: bi-infinite discrete Laplacian}
The bi-infinite discrete Laplacian arises in simple models of crystal structure, though we can also think of it as a finite difference discretisation of the Laplacian on all of $\R$.  That is, consider the bi-infinite continuous Helmholtz equation

\[
u''(x) - k^2 u(x) = f(x)
\]
for $x \in \R$. If we represent the second derivative by its finite-difference approximation at the grid $x_k = k h$,

\[
{u(x_{k-1}) -2 u(x_k) + u(x_{k+1}) \over h^2}
\]
That is, we have the discretisation

\[
\Delta_h := {1 \over h^2}L[z-2+z^{-1}] = {1 \over h^2} \sopmatrix{\ddots & \ddots \\ 
                \ddots & -2 & 1 \\
                & 1 & -2 & 1 \\
                && 1 & -2 & \ddots \\
                &&&\ddots & \ddots}
\]
which gives us the system:

\[
(\Delta_h - k^2 I) \vc u = \vc f
\]
Noting that

\[
\Delta_h - k^2 I  = L[\underbrace{z/h^2  - k^2 - 2/h^2 + z^{-1}/h^2}_{a(z)}]
\]
we know that we can invert this exactly as $L[a(z)^{-1}]$. To determite its Laurent series  we do partial fraction expansion

\[
a(z)^{-1} = \br[{(z-z_-) (z - z_+) \over  z h^2} ]^{-1} = 
{z_- h^2 \over (z-z_-) (z_--z_+)}  + {z_+ h^2 \over (z-z_+) (z_+-z_-)}
\]
where 

\[
z_\pm = 1 + {h^2k^2 \over 2} \pm {h k \over 2} \sqrt{h^2 k^2+4}.
\]
To find the Laurent series we use $|z_-| < 1$ so that

\[
{1 \over z - z_-} = {1 \over z} + {z_- \over z^2} + {z_-^2 \over z^3} + \cdots
\]
and $|z_+| > 1$ so that

\[
{1 \over z - z_+} = -{1 \over z_+} - {z \over z_+^2} - {z^2 \over z_+^3} + \cdots,
\]
hence

\[
a(z)^{-1} = {z_-^2 h^2 \over z_- - z_+} \sum_{k=-\infty}^{-1} z_-^{-k} z^k + {h^2 \over z_- - z_+} \sum_{k=0}^\infty z_+^k z^k
\]
The entries of the Laurent series then give the entries of $L[a(z)^{-1}]$.

\subsection{Toeplitz operators}
A \emph{Toeplitz operator} is an infinite-dimensional matrix with constant diagonals. As in a Laurent operator, we associate it with a Laurent series on the unit circle called the \emph{symbol}:

\[
a(z) = \sum_{k=-\infty}^\infty a_k z^k
\]
Then the corresponding Toeplitz operator is

\[
T[a(z)] := \sopmatrix{
 a_0 & a_{-1} & a_{-2} & a_{-3} & \ddots \\
 a_1 & a_0 & a_{-1} & a_{-2} & \ddots \\
 a_2 & a_1 & a_0 & a_{-1}   & \ddots \\
\vdots & \ddots& \ddots& \ddots & \ddots
}
\]
In other words, for $k,j \in \{0,1,2,3,\dots \}$ we have

\[
\vc e_k^\top T[a(z)] \vc e_j = a_{k-j}
\]
Note that $T[a(z)] \vc b$ is a half-discrete convolution:

\[
\vc e_k^\top L[a(z)] \vc b = \sum_{j=0}^\infty a_{k-j} b_j
\]
\subsubsection{Algebraic properties}
Unlike Laurent operators, we do not have a trivial algebraic structure: in general

\[
T[a(z)] T[b(z)] \neq T[a(z) b(z)] \neq T[b(z)] T[a(z)]
\]
For example, if $a(z) = z$ and $b(z) = z^{-1}$ so that $T[a(z)b(z)] = I$ we have

\[
\underbrace{\sopmatrix{0 \\ 1 \\ & 1 \\ &&\ddots}}_{T[a(z)]} \underbrace{\sopmatrix{0 & 1 \\ & & 1 \\ &&&\ddots}}_{T[b(z)]} = \sopmatrix{0 \\ & 1 \\ && 1 \\ &&& \ddots} \neq I
\]
On the other hand $T[b(z)] T[a(z)] = I$, and we will see that the analyticity properties  are what ensure this, and thence lead to a RH problem. 

The easiest way to understand the algebraic properties is to note that a Toeplitz operator is  a sub-component of a Laurent operator, that is,

\[
L[a(z)] = \sopmatrix{
\times & \times \\ \times & T[a(z)]
}
\]
(Here $\times$ means we don't use those entries.) In the special case where 

\[
a(z) = \sum_{k=0}^\infty a_k z^k \qqand b(z) = \sum_{k=0}^\infty b_k z^k
\]
are analytic, $L[a(z)]$, $L[b(z)]$, and $L[a(z) b(z)]$ are lower-triangular, hence


\begin{align*}
\sopmatrix{
\times &  \\ \times & T[a(z) b(z)]
} &= L[a(z)b(z)] = L[a(z)] L[b(z)] =  \sopmatrix{
\times &  \\ \times & T[a(z)]
} \sopmatrix{
\times &  \\ \times & T[b(z)]
} \ccr
= \sopmatrix{
\times &  \\ \times & T[a(z)] T[b(z)]
}
\end{align*}
Similarly, if 

\[
a(z) = \sum_{k=-\infty}^0 a_k z^k \qqand b(z) = \sum_{k=-\infty}^0 b_k z^k
\]
are analytic outside the unit circle,  $L[a(z)]$, $L[b(z)]$, and $L[a(z) b(z)]$ are upper-triangular, hence


\begin{align*}
\sopmatrix{
\times & \times \\  & T[a(z) b(z)]
} &= L[a(z)b(z)] = L[a(z)] L[b(z)] =  \sopmatrix{
\times & \times \\  & T[a(z)]
} \sopmatrix{
\times & \times  \\  & T[b(z)]
} \ccr
= \sopmatrix{
\times & \times  \\  & T[a(z)] T[b(z)]
}
\end{align*}
In both cases, if $a(z)$ is analytic inside/outside and has no roots, then the above imply that 

\[
T[a(z)]^{-1} = T[a(z)^{-1}]
\]
Finally, if 

\[
a(z) = \sum_{k=-\infty}^0 a_k z^k \qqand b(z) = \sum_{k=0}^\infty b_k z^k
\]
that is $a(z)$ is analytic outside and $b(z)$ is analytic inside the unit circle, then we have $L[a(z)]$ is upper triangular, $L[b(z)]$ is lower triangular, and


\begin{align*}
\sopmatrix{
\times & \times \\ \times & T[a(z) b(z)]
} &= L[a(z)b(z)] = L[a(z)] L[b(z)] =  \sopmatrix{
\times & \times \\  & T[a(z)]
} \sopmatrix{
\times &   \\ \times & T[b(z)]
} \ccr
= \sopmatrix{
\times & \times  \\ \times & T[a(z)] T[b(z)]
}
\end{align*}
\subsubsection{Wiener-Hopf factorisation}
This presents a procedure for inverting Toeplitz operators: find $\phi_+(z)$ analytic inside the unit circle, and $\phi_-(z)$ is analytic outside with $\phi_-(\infty) = 1$ satisfying

\[
a(z) = \phi_+(z) \phi_-(z)
\]
so that we have the UL decomposition

\[
T[a(z)] = T[\phi_-(z)] T[\phi_+(z)]
\]
This is equivalent to solving the RH problem

\[
\phi_+(z) = a(z) \phi_-(z)^{-1}
\]
assuming $\phi_-(z)$ has no zeros. As outlined last lecture, this is satisfied if the winding number of $a(z)$ is trivial. In particular, we have the solution

\[
\phi_\pm(z) = \E^{\pm \CC[\log a](z)}
\]
though when $a(z)$ is rational it is more feasible to compute the roots and factorise appropriately.

\subsubsection{Example: Finite difference on half-line}
We consider a semi-infinite discrete Laplacian. This arises naturally from a finite-difference discretisation with zero Dirichlet conditions, e.g., for $x > 0$ we have

\[
u''(x)  - k^2 u(x) = f(x) \qqand u(0) = 0
\]
We discretise at $x_k = h k$ for $k = 0,1,\ldots$. The finite-difference gives us 


\begin{align*}
u''(x_1) \approx {u(x_0) - 2 u(x_1) + u(x_2) \over h^2} &=  {-2 u(x_1) + u(x_2) \over h^2} \cr
u''(x_k) \approx {u(x_{k-1}) - 2 u(x_k) + u(x_{k+1}) \over h^2} 
\end{align*}
That is, we have the discretisation

\[
\Delta_h := {1 \over h^2}T[z-2+z^{-1}] = {1 \over h^2} \sopmatrix{
                 -2 & 1 \\
                 1 & -2 & 1 \\
                & 1 & -2 & \ddots \\
                &&\ddots & \ddots}
\]
which gives us the system:

\[
(\Delta_h - k^2 I) \vc u = \vc f
\]
where 

\[
\vc u = \Vectt[u(x_1),u(x_2),\dots],\qquad \vc f = \Vectt[f(x_1),f(x_2),\dots].
\]
Noting that

\[
\Delta_h - k^2 I  = T[\underbrace{z/h^2  - k^2 - 2/h^2 + z^{-1}/h^2}_{a(z)}]
\]
we know that we can invert this exactly by finding the Wiener\ensuremath{\endash}Hopf factorisation of $a(z)$, namely,

\[
a(z) = \underbrace{(1-z_-/z)}_{\phi_-(z)} \underbrace{h^{-1} (z-z_+)}_{\phi_+(z)}
\]
where $z_\pm$ were defined above. Therefore

\[
T[a(z)] = T[\phi_-(z)] T[\phi_+(z)]
\]
This implies that

\[
T[a(z)]^{-1} = T[\phi_+(z)]^{-1} T[\phi_-(z)]^{-1} = T[\phi_+(z)^{-1}] T[\phi_-(z)^{-1}]
\]
We can use Geometric series to compute 


\begin{align*}
\phi_+(z)^{-1} &= {h \over z - z_+} = -{h \over z_+} -{h z \over z_+^2} - {h z^2 \over z_+^3} - \cdots \ccr
\phi_-(z)^{-1} = {1 \over 1-z_-/z} = 1 + {z_- \over z} + {z_-^2 \over z^2} + \cdots
\end{align*}
The Laurent coefficients tell us the entries of $T[\phi_\pm(z)^{-1}]$.

\textbf{Demonstration} We want to factorise and invert $T[a(z)]$:


\begin{lstlisting}
(*@\HLJLk{using}@*) (*@\HLJLn{ToeplitzMatrices}@*)(*@\HLJLp{,}@*) (*@\HLJLn{LinearAlgebra}@*)
(*@\HLJLn{h}@*) (*@\HLJLoB{=}@*) (*@\HLJLn{k}@*) (*@\HLJLoB{=}@*) (*@\HLJLni{1}@*)
(*@\HLJLn{n}@*) (*@\HLJLoB{=}@*) (*@\HLJLni{10}@*)
(*@\HLJLn{\ensuremath{\Delta}}@*) (*@\HLJLoB{=}@*) (*@\HLJLni{1}@*)(*@\HLJLoB{/}@*)(*@\HLJLn{h}@*)(*@\HLJLoB{{\textasciicircum}}@*)(*@\HLJLni{2}@*) (*@\HLJLoB{*}@*) (*@\HLJLnf{Toeplitz}@*)(*@\HLJLp{([}@*)(*@\HLJLoB{-}@*)(*@\HLJLni{2}@*)(*@\HLJLp{;}@*) (*@\HLJLni{1}@*)(*@\HLJLp{;}@*) (*@\HLJLnf{zeros}@*)(*@\HLJLp{(}@*)(*@\HLJLn{n}@*)(*@\HLJLoB{-}@*)(*@\HLJLni{2}@*)(*@\HLJLp{)],}@*) (*@\HLJLp{[}@*)(*@\HLJLoB{-}@*)(*@\HLJLni{2}@*)(*@\HLJLp{;}@*) (*@\HLJLni{1}@*)(*@\HLJLp{;}@*) (*@\HLJLnf{zeros}@*)(*@\HLJLp{(}@*)(*@\HLJLn{n}@*)(*@\HLJLoB{-}@*)(*@\HLJLni{2}@*)(*@\HLJLp{)])}@*)
(*@\HLJLn{T}@*) (*@\HLJLoB{=}@*) (*@\HLJLn{\ensuremath{\Delta}}@*) (*@\HLJLoB{-}@*) (*@\HLJLn{k}@*)(*@\HLJLoB{{\textasciicircum}}@*)(*@\HLJLni{2}@*) (*@\HLJLoB{*}@*) (*@\HLJLn{I}@*)
\end{lstlisting}

\begin{lstlisting}
10(*@\ensuremath{\times}@*)10 Array(*@{{\{}}@*)Float64,2(*@{{\}}}@*):
 -3.0   1.0   0.0   0.0   0.0   0.0   0.0   0.0   0.0   0.0
  1.0  -3.0   1.0   0.0   0.0   0.0   0.0   0.0   0.0   0.0
  0.0   1.0  -3.0   1.0   0.0   0.0   0.0   0.0   0.0   0.0
  0.0   0.0   1.0  -3.0   1.0   0.0   0.0   0.0   0.0   0.0
  0.0   0.0   0.0   1.0  -3.0   1.0   0.0   0.0   0.0   0.0
  0.0   0.0   0.0   0.0   1.0  -3.0   1.0   0.0   0.0   0.0
  0.0   0.0   0.0   0.0   0.0   1.0  -3.0   1.0   0.0   0.0
  0.0   0.0   0.0   0.0   0.0   0.0   1.0  -3.0   1.0   0.0
  0.0   0.0   0.0   0.0   0.0   0.0   0.0   1.0  -3.0   1.0
  0.0   0.0   0.0   0.0   0.0   0.0   0.0   0.0   1.0  -3.0
\end{lstlisting}


We can find the UL decomposition as follows


\begin{lstlisting}
(*@\HLJLn{z{\_}p}@*) (*@\HLJLoB{=}@*) (*@\HLJLni{1}@*) (*@\HLJLoB{+}@*) (*@\HLJLp{(}@*)(*@\HLJLn{h}@*)(*@\HLJLoB{{\textasciicircum}}@*)(*@\HLJLni{2}@*) (*@\HLJLoB{*}@*) (*@\HLJLn{k}@*)(*@\HLJLoB{{\textasciicircum}}@*)(*@\HLJLni{2}@*)(*@\HLJLp{)}@*)(*@\HLJLoB{/}@*)(*@\HLJLni{2}@*) (*@\HLJLoB{+}@*) (*@\HLJLp{(}@*)(*@\HLJLn{h}@*)(*@\HLJLoB{*}@*)(*@\HLJLn{k}@*)(*@\HLJLp{)}@*)(*@\HLJLoB{/}@*)(*@\HLJLni{2}@*) (*@\HLJLoB{*}@*) (*@\HLJLnf{sqrt}@*)(*@\HLJLp{(}@*)(*@\HLJLn{h}@*)(*@\HLJLoB{{\textasciicircum}}@*)(*@\HLJLni{2}@*)(*@\HLJLoB{*}@*)(*@\HLJLn{k}@*)(*@\HLJLoB{{\textasciicircum}}@*)(*@\HLJLni{2}@*)(*@\HLJLoB{+}@*)(*@\HLJLni{4}@*)(*@\HLJLp{)}@*)
(*@\HLJLn{z{\_}m}@*) (*@\HLJLoB{=}@*) (*@\HLJLni{1}@*) (*@\HLJLoB{+}@*) (*@\HLJLp{(}@*)(*@\HLJLn{h}@*)(*@\HLJLoB{{\textasciicircum}}@*)(*@\HLJLni{2}@*) (*@\HLJLoB{*}@*) (*@\HLJLn{k}@*)(*@\HLJLoB{{\textasciicircum}}@*)(*@\HLJLni{2}@*)(*@\HLJLp{)}@*)(*@\HLJLoB{/}@*)(*@\HLJLni{2}@*) (*@\HLJLoB{-}@*) (*@\HLJLp{(}@*)(*@\HLJLn{h}@*)(*@\HLJLoB{*}@*)(*@\HLJLn{k}@*)(*@\HLJLp{)}@*)(*@\HLJLoB{/}@*)(*@\HLJLni{2}@*) (*@\HLJLoB{*}@*) (*@\HLJLnf{sqrt}@*)(*@\HLJLp{(}@*)(*@\HLJLn{h}@*)(*@\HLJLoB{{\textasciicircum}}@*)(*@\HLJLni{2}@*)(*@\HLJLoB{*}@*)(*@\HLJLn{k}@*)(*@\HLJLoB{{\textasciicircum}}@*)(*@\HLJLni{2}@*)(*@\HLJLoB{+}@*)(*@\HLJLni{4}@*)(*@\HLJLp{)}@*)

(*@\HLJLn{z}@*) (*@\HLJLoB{=}@*) (*@\HLJLnf{exp}@*)(*@\HLJLp{(}@*)(*@\HLJLnfB{0.1}@*)(*@\HLJLn{im}@*)(*@\HLJLp{)}@*)
(*@\HLJLp{(}@*)(*@\HLJLni{1}@*)(*@\HLJLoB{-}@*)(*@\HLJLn{z{\_}p}@*)(*@\HLJLoB{/}@*)(*@\HLJLn{z}@*)(*@\HLJLp{)}@*)(*@\HLJLoB{*}@*)(*@\HLJLp{(}@*)(*@\HLJLnf{inv}@*)(*@\HLJLp{(}@*)(*@\HLJLn{h}@*)(*@\HLJLp{)}@*) (*@\HLJLoB{*}@*) (*@\HLJLp{(}@*)(*@\HLJLn{z}@*)(*@\HLJLoB{-}@*)(*@\HLJLn{z{\_}m}@*)(*@\HLJLp{))}@*)
(*@\HLJLn{z}@*)(*@\HLJLoB{/}@*)(*@\HLJLn{h}@*)(*@\HLJLoB{{\textasciicircum}}@*)(*@\HLJLni{2}@*) (*@\HLJLoB{-}@*) (*@\HLJLni{2}@*)(*@\HLJLoB{/}@*)(*@\HLJLn{h}@*)(*@\HLJLoB{{\textasciicircum}}@*)(*@\HLJLni{2}@*) (*@\HLJLoB{+}@*) (*@\HLJLni{1}@*)(*@\HLJLoB{/}@*)(*@\HLJLp{(}@*)(*@\HLJLn{z}@*)(*@\HLJLoB{*}@*)(*@\HLJLn{h}@*)(*@\HLJLoB{{\textasciicircum}}@*)(*@\HLJLni{2}@*)(*@\HLJLp{)}@*) (*@\HLJLoB{-}@*) (*@\HLJLn{k}@*)(*@\HLJLoB{{\textasciicircum}}@*)(*@\HLJLni{2}@*)

(*@\HLJLn{U}@*) (*@\HLJLoB{=}@*) (*@\HLJLnf{Toeplitz}@*)(*@\HLJLp{([}@*)(*@\HLJLni{1}@*)(*@\HLJLp{;}@*) (*@\HLJLnf{zeros}@*)(*@\HLJLp{(}@*)(*@\HLJLn{n}@*)(*@\HLJLoB{-}@*)(*@\HLJLni{1}@*)(*@\HLJLp{)],}@*) (*@\HLJLp{[}@*)(*@\HLJLni{1}@*)(*@\HLJLp{;}@*) (*@\HLJLoB{-}@*)(*@\HLJLn{z{\_}m}@*)(*@\HLJLp{;}@*) (*@\HLJLnf{zeros}@*)(*@\HLJLp{(}@*)(*@\HLJLn{n}@*)(*@\HLJLoB{-}@*)(*@\HLJLni{2}@*)(*@\HLJLp{)])}@*)
\end{lstlisting}

\begin{lstlisting}
10(*@\ensuremath{\times}@*)10 Toeplitz(*@{{\{}}@*)Float64,Complex(*@{{\{}}@*)Float64(*@{{\}}}@*)(*@{{\}}}@*):
 1.0  -0.381966   0.0        0.0       (*@\ensuremath{\dots}@*)   0.0        0.0        0.0     
 0.0   1.0       -0.381966   0.0           0.0        0.0        0.0     
 0.0   0.0        1.0       -0.381966      0.0        0.0        0.0     
 0.0   0.0        0.0        1.0           0.0        0.0        0.0     
 0.0   0.0        0.0        0.0           0.0        0.0        0.0     
 0.0   0.0        0.0        0.0       (*@\ensuremath{\dots}@*)   0.0        0.0        0.0     
 0.0   0.0        0.0        0.0          -0.381966   0.0        0.0     
 0.0   0.0        0.0        0.0           1.0       -0.381966   0.0     
 0.0   0.0        0.0        0.0           0.0        1.0       -0.381966
 0.0   0.0        0.0        0.0           0.0        0.0        1.0
\end{lstlisting}


\begin{lstlisting}
(*@\HLJLn{L}@*) (*@\HLJLoB{=}@*) (*@\HLJLnf{Toeplitz}@*)(*@\HLJLp{([}@*)(*@\HLJLoB{-}@*)(*@\HLJLn{z{\_}p}@*)(*@\HLJLoB{/}@*)(*@\HLJLn{h}@*)(*@\HLJLp{;}@*) (*@\HLJLni{1}@*)(*@\HLJLoB{/}@*)(*@\HLJLn{h}@*)(*@\HLJLp{;}@*) (*@\HLJLnf{zeros}@*)(*@\HLJLp{(}@*)(*@\HLJLn{n}@*)(*@\HLJLoB{-}@*)(*@\HLJLni{2}@*)(*@\HLJLp{)],}@*) (*@\HLJLp{[}@*)(*@\HLJLoB{-}@*)(*@\HLJLn{z{\_}p}@*)(*@\HLJLoB{/}@*)(*@\HLJLn{h}@*)(*@\HLJLp{;}@*) (*@\HLJLnf{zeros}@*)(*@\HLJLp{(}@*)(*@\HLJLn{n}@*)(*@\HLJLoB{-}@*)(*@\HLJLni{1}@*)(*@\HLJLp{)])}@*)
\end{lstlisting}

\begin{lstlisting}
10(*@\ensuremath{\times}@*)10 Toeplitz(*@{{\{}}@*)Float64,Complex(*@{{\{}}@*)Float64(*@{{\}}}@*)(*@{{\}}}@*):
 -2.61803   0.0       0.0       0.0      (*@\ensuremath{\dots}@*)   0.0       0.0       0.0    
  1.0      -2.61803   0.0       0.0          0.0       0.0       0.0    
  0.0       1.0      -2.61803   0.0          0.0       0.0       0.0    
  0.0       0.0       1.0      -2.61803      0.0       0.0       0.0    
  0.0       0.0       0.0       1.0          0.0       0.0       0.0    
  0.0       0.0       0.0       0.0      (*@\ensuremath{\dots}@*)   0.0       0.0       0.0    
  0.0       0.0       0.0       0.0          0.0       0.0       0.0    
  0.0       0.0       0.0       0.0         -2.61803   0.0       0.0    
  0.0       0.0       0.0       0.0          1.0      -2.61803   0.0    
  0.0       0.0       0.0       0.0          0.0       1.0      -2.61803
\end{lstlisting}


\begin{lstlisting}
(*@\HLJLn{U}@*)(*@\HLJLoB{*}@*)(*@\HLJLn{L}@*)
\end{lstlisting}

\begin{lstlisting}
10(*@\ensuremath{\times}@*)10 Array(*@{{\{}}@*)Float64,2(*@{{\}}}@*):
 -3.0   1.0   0.0   0.0   0.0   0.0   0.0   0.0   0.0   0.0    
  1.0  -3.0   1.0   0.0   0.0   0.0   0.0   0.0   0.0   0.0    
  0.0   1.0  -3.0   1.0   0.0   0.0   0.0   0.0   0.0   0.0    
  0.0   0.0   1.0  -3.0   1.0   0.0   0.0   0.0   0.0   0.0    
  0.0   0.0   0.0   1.0  -3.0   1.0   0.0   0.0   0.0   0.0    
  0.0   0.0   0.0   0.0   1.0  -3.0   1.0   0.0   0.0   0.0    
  0.0   0.0   0.0   0.0   0.0   1.0  -3.0   1.0   0.0   0.0    
  0.0   0.0   0.0   0.0   0.0   0.0   1.0  -3.0   1.0   0.0    
  0.0   0.0   0.0   0.0   0.0   0.0   0.0   1.0  -3.0   1.0    
  0.0   0.0   0.0   0.0   0.0   0.0   0.0   0.0   1.0  -2.61803
\end{lstlisting}


This equals \texttt{T}, apart from the bottom right entry, which is off as these are finite-dimensional matrices. 

We can then invert \texttt{U} and \texttt{L}, to construct the inverse of \texttt{T}:


\begin{lstlisting}
(*@\HLJLn{Ui}@*) (*@\HLJLoB{=}@*) (*@\HLJLnf{Toeplitz}@*)(*@\HLJLp{([}@*)(*@\HLJLni{1}@*)(*@\HLJLp{;}@*) (*@\HLJLnf{zeros}@*)(*@\HLJLp{(}@*)(*@\HLJLn{n}@*)(*@\HLJLoB{-}@*)(*@\HLJLni{1}@*)(*@\HLJLp{)],}@*) (*@\HLJLn{z{\_}m}@*)(*@\HLJLoB{.{\textasciicircum}}@*)(*@\HLJLp{(}@*)(*@\HLJLni{0}@*)(*@\HLJLoB{:}@*)(*@\HLJLn{n}@*)(*@\HLJLoB{-}@*)(*@\HLJLni{1}@*)(*@\HLJLp{))}@*)
(*@\HLJLn{Li}@*) (*@\HLJLoB{=}@*) (*@\HLJLoB{-}@*)(*@\HLJLn{h}@*)(*@\HLJLoB{*}@*)(*@\HLJLnf{Toeplitz}@*)(*@\HLJLp{(}@*)(*@\HLJLn{z{\_}p}@*)(*@\HLJLoB{.{\textasciicircum}}@*)(*@\HLJLp{(}@*)(*@\HLJLoB{-}@*)(*@\HLJLp{(}@*)(*@\HLJLni{1}@*)(*@\HLJLoB{:}@*)(*@\HLJLn{n}@*)(*@\HLJLp{)),}@*) (*@\HLJLp{[}@*)(*@\HLJLni{1}@*)(*@\HLJLoB{/}@*)(*@\HLJLn{z{\_}p}@*)(*@\HLJLp{;}@*) (*@\HLJLnf{zeros}@*)(*@\HLJLp{(}@*)(*@\HLJLn{n}@*)(*@\HLJLoB{-}@*)(*@\HLJLni{1}@*)(*@\HLJLp{)])}@*)
(*@\HLJLn{Ti}@*) (*@\HLJLoB{=}@*) (*@\HLJLn{Li}@*)(*@\HLJLoB{*}@*)(*@\HLJLn{Ui}@*)
\end{lstlisting}

\begin{lstlisting}
10(*@\ensuremath{\times}@*)10 Array(*@{{\{}}@*)Float64,2(*@{{\}}}@*):
 -0.381966     -0.145898     -0.0557281    (*@\ensuremath{\dots}@*)  -0.00017307   -6.6107e-5  
 -0.145898     -0.437694     -0.167184        -0.000519211  -0.000198321
 -0.0557281    -0.167184     -0.445825        -0.00138456   -0.000528856
 -0.0212862    -0.0638587    -0.17029         -0.00363448   -0.00138825 
 -0.00813062   -0.0243919    -0.065045        -0.00951886   -0.00363588 
 -0.00310562   -0.00931686   -0.024845     (*@\ensuremath{\dots}@*)  -0.0249221    -0.0095194  
 -0.00118624   -0.00355872   -0.00948993      -0.0652475    -0.0249223  
 -0.000453104  -0.00135931   -0.00362483      -0.17082      -0.0652476  
 -0.00017307   -0.000519211  -0.00138456      -0.447214     -0.17082    
 -6.6107e-5    -0.000198321  -0.000528856     -0.17082      -0.447214
\end{lstlisting}


This is an inverse (apart from truncation error)


\begin{lstlisting}
(*@\HLJLnf{norm}@*)(*@\HLJLp{((}@*)(*@\HLJLn{Ti}@*)(*@\HLJLoB{*}@*)(*@\HLJLn{T}@*) (*@\HLJLoB{-}@*) (*@\HLJLn{I}@*)(*@\HLJLp{)[}@*)(*@\HLJLni{1}@*)(*@\HLJLoB{:}@*)(*@\HLJLk{end}@*)(*@\HLJLoB{-}@*)(*@\HLJLni{1}@*)(*@\HLJLp{,}@*)(*@\HLJLni{1}@*)(*@\HLJLoB{:}@*)(*@\HLJLk{end}@*)(*@\HLJLoB{-}@*)(*@\HLJLni{1}@*)(*@\HLJLp{])}@*)
\end{lstlisting}

\begin{lstlisting}
4.2276150673973256e-16
\end{lstlisting}



\end{document}
